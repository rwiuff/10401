
\subsection{Assignment 1: A simple reactor model}

Freidberg's model produced some reasonable results for a simple tokamak design. The model would however break down when certain parameters would lie in certain regions. For example $\beta$ would go to infinity when $P_{\mathrm{E}}$ would lie approximately between \SI{500}{\mega\watt} and \SI{1000}{\mega\watt}. Meanwhile after implementing the elliptical model and solving it for $P_{\mathrm{E}}=\SI{2}{\giga\watt}$ $\beta$ would go to $1.69\%$ which we deemed to low. Setting $\kappa=2$ and $A=3$ would increase $\beta$ to $4.55\%$. Further optimising $\beta$ and $V_{\mathrm{P}}/A_{\mathrm{P}}$ would lead to $\kappa=2.5$ and $A=2.39$ using a desired $\beta=10\%$.

\subsection{Assignment 2: Diagnostics via interferometry}
\todo{Denne sektion skal nok kigges igennem}

Wanting to keep the emitter frequency above the plasma frequency resulted in a minimum frequency of \SI{9}{\giga\hertz} given the calculations, this is the optimum frequency however the best emitter available is emitting at a frequency of \SI{60}{\giga\hertz} so this emitter should be chosen. In order to make sure that the beam is controlled and that most of the beam is transmitted, a Gaussian telescope with a focal length of \SI{0.250}{\meter} is implemented. It is also important that the beam hits the plasma surface perpendicular so refraction is avoided.

\subsection{Assignment 3: Fusor exercise}
While it is hard to conclude anything based on the very poor measurements, minor conclusions are somewhat possible. Barring the trivial measurement of the plasma constituents and focusing on the information the spectral line width should provide. While(again) measurements are very uncertain it does seem, as expected, that it measures that the velocity of the deuterium is smaller than for the hydrogen at a given voltage. This points towards the measurement being correct and the average ion energy as a function of voltage can then be calculated. While an increasing energy was derived with increasing voltage, the result did not correspond with expectations as the energies for the two species were not equal. On the neutron rates. A very optimistic conclusion of increasing neutron rates with voltage and current was achieved. The more important conclusion for now seems to be that an automatic pressure regulator must be implemented if an experiment is to be precise. Furthermore, a much larger amount of data points must also be gathered.