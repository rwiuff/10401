% !TEX root = ../Main.tex
Studying the Inertial electrostatic confinement fusor at DTU. Measurements of neutron counts and the spectral line width are made as a function of the voltage and current. To this, the emission spectrum of the gas are also measured.
\subsection{Plasma light emission and spectral line}
When doing spectroscopy on a plasma of an unknown gas. Optical dispersion splits up the spectrum of light into lines which are then recorded using a detector. This detector measures the frequency/wavelength of the light and the intensity of the light. Thus a plot with the wavelength along the x-axis vs the intensity along the y-axis are produced. For our recordings a typical spectrum looked like that of \cref{SPEC}.
\begin{figure}[H]
	\centering
	\resizebox{\textwidth}{!}{
		\begin{tikzpicture}
			\begin{axis}[
					width=\textwidth,
          height=\axisdefaultheight,
					title={Spectrum of light from fusion in the fusor},
					use units,
					x unit prefix=n, x unit=m,
					y unit=A.U.,
					xlabel=Wavelength,
					ylabel=Intensity,
					ymajorgrids=true,
					grid style=dashed,
					%scaled y ticks=manual:{\(10^{-9}\)}{\pgfmathparse{#1*10^9}}
				]
				\addplot[color=red] table {Data/SpectrumData.txt};
			\end{axis}
		\end{tikzpicture}
	}
	\caption{Spectrum of light from fusion in the fusor}
	\label{SPEC}
\end{figure}
Each peak correspond with a constituent gas in the fusor.
The peaks present on the spectrum are subject to broadening effects. Considering the Doppler effect, each photon's frequency will be shifted since the particles (which are the emitters) are moving at fast speeds (the Doppler shift increases with speed). Therefore if each particle where moving in the same direction the entire intensity peak would be shifted. But since the particles are moving more or less isotropic around the centre of the device, the peaks are not shifted but rather broadened around the normal emission wavelength. To this it is also noted that the Doppler broadening increases if particles are moving faster.\\
Considering \cref{fig:Spectro}, a spectral line is shown with a central wavelength of around \SI{656}{\nano\meter} which correspond to an accepted value of hydrogen light emission\footnote{\href{https://en.wikipedia.org/wiki/Hydrogen_spectral_series}{``Hydrogen spectral series'' - Wikipedia (Visited January \nth{22})}}. Meanwhile the spectral lines of Deuterium only differs by a factor of $1.000272$\footnote{\href{https://en.wikipedia.org/wiki/Deuterium#Spectroscopy}{``Deuterium\#Spectroscopy'' - Wikipedia (Visited January \nth{22})}}, which will not be detected using our methods. Two much smaller peaks at \SI{486}{\nano\meter} and \SI{433}{\nano\meter} where also found. These are also in agreement where the \SI{433}{\nano\meter} is of by \SI{1}{\nano\meter}. This is due to uncertainties, as each wavelength is found using a Gaussian fit.\\
Meanwhile the spectral line width is expected to increase with electric potential. This is because that increasing the potential between the electrodes means increasing the electric field which in turn increases the acceleration on the ions due to the Lorentz force. This means that more particles will be accelerated to higher speeds which increases the Doppler broadening around the normal wavelength.
\begin{figure}[ht!]
	\centering
	\includegraphics[width=\textwidth]{Figures/D_linje.png}
	\caption{Spectral line for deuterium measured in the fusor.}
	\label{fig:Spectro}
\end{figure}
\subsection{Neutron production rate}
The fusion reaction rate should increase with voltage and current. Assuming that the reaction rate increases linearly with current, the largest effect is seen from the voltage increase.
